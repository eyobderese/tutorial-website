\documentclass{article}
\usepackage{amsmath}
\usepackage{amssymb}
\usepackage{graphicx}
\usepackage{listings}
\usepackage{hyperref}

\title{Calculus Fundamentals}
\author{TutorialHub Team}
\date{June 10, 2023}
\keywords{Calculus, Derivatives, Integrals, Limits}
\category{Mathematics}

\begin{document}

\maketitle

\begin{abstract}
This tutorial covers the fundamental concepts of calculus, including limits, derivatives, and integrals. We'll explore key theorems and practical applications.
\end{abstract}

\section{Introduction to Calculus}

Calculus is the mathematical study of continuous change. It has two major branches: differential calculus (concerning rates of change and slopes of curves) and integral calculus (concerning accumulation of quantities and the areas under curves).

\section{Limits}

The concept of a limit is fundamental to calculus. Informally, the limit of a function at a point is the value that the function approaches as its input approaches that point.

\subsection{Definition of a Limit}

We say that the limit of $f(x)$ as $x$ approaches $a$ is $L$, written as:

$$\lim_{x \to a} f(x) = L$$

if for every $\epsilon > 0$, there exists a $\delta > 0$ such that:

$$0 < |x - a| < \delta \implies |f(x) - L| < \epsilon$$

\subsection{Properties of Limits}

For functions $f$ and $g$ and constants $a$ and $c$:

\begin{align}
\lim_{x \to a} [f(x) + g(x)] &= \lim_{x \to a} f(x) + \lim_{x \to a} g(x) \\
\lim_{x \to a} [f(x) \cdot g(x)] &= \lim_{x \to a} f(x) \cdot \lim_{x \to a} g(x) \\
\lim_{x \to a} \frac{f(x)}{g(x)} &= \frac{\lim_{x \to a} f(x)}{\lim_{x \to a} g(x)}, \text{ if } \lim_{x \to a} g(x) \neq 0 \\
\lim_{x \to a} c \cdot f(x) &= c \cdot \lim_{x \to a} f(x)
\end{align}

\section{Derivatives}

The derivative of a function represents its rate of change. Geometrically, it represents the slope of the tangent line to the function's graph at a given point.

\subsection{Definition of a Derivative}

The derivative of a function $f(x)$ with respect to $x$ is defined as:

$$f'(x) = \lim_{h \to 0} \frac{f(x + h) - f(x)}{h}$$

\subsection{Common Derivative Rules}

\begin{align}
\frac{d}{dx}(c) &= 0 \text{ (constant rule)} \\
\frac{d}{dx}(x^n) &= nx^{n-1} \text{ (power rule)} \\
\frac{d}{dx}(e^x) &= e^x \text{ (exponential rule)} \\
\frac{d}{dx}(\ln x) &= \frac{1}{x} \text{ (logarithm rule)} \\
\frac{d}{dx}(\sin x) &= \cos x \text{ (sine rule)} \\
\frac{d}{dx}(\cos x) &= -\sin x \text{ (cosine rule)}
\end{align}

\subsection{Product Rule}

If $f$ and $g$ are differentiable functions, then:

$$\frac{d}{dx}[f(x) \cdot g(x)] = f'(x) \cdot g(x) + f(x) \cdot g'(x)$$

\subsection{Chain Rule}

If $f$ and $g$ are differentiable functions and $y = f(g(x))$, then:

$$\frac{dy}{dx} = \frac{df}{dg} \cdot \frac{dg}{dx}$$

\section{Integrals}

Integration is the reverse process of differentiation. It allows us to find the accumulated value of a function over an interval.

\subsection{Indefinite Integrals}

The indefinite integral of a function $f(x)$ is written as:

$$\int f(x) \, dx = F(x) + C$$

where $F'(x) = f(x)$ and $C$ is an arbitrary constant.

\subsection{Definite Integrals}

The definite integral of a function $f(x)$ from $a$ to $b$ is defined as:

$$\int_a^b f(x) \, dx = F(b) - F(a)$$

where $F'(x) = f(x)$.

\subsection{Fundamental Theorem of Calculus}

The Fundamental Theorem of Calculus connects differentiation and integration:

If $f$ is continuous on $[a, b]$ and $F$ is an antiderivative of $f$, then:

$$\int_a^b f(x) \, dx = F(b) - F(a)$$

\subsection{Common Integration Rules}

\begin{align}
\int x^n \, dx &= \frac{x^{n+1}}{n+1} + C, \text{ for } n \neq -1 \\
\int e^x \, dx &= e^x + C \\
\int \frac{1}{x} \, dx &= \ln|x| + C \\
\int \sin x \, dx &= -\cos x + C \\
\int \cos x \, dx &= \sin x + C
\end{align}

\section{Applications of Calculus}

Calculus has numerous applications in physics, engineering, economics, and other fields.

\subsection{Finding Maxima and Minima}

To find the maximum or minimum values of a function:
\begin{enumerate}
  \item Find the critical points by solving $f'(x) = 0$
  \item Use the second derivative test: if $f''(x) < 0$ at a critical point, it's a maximum; if $f''(x) > 0$, it's a minimum
\end{enumerate}

\subsection{Area Between Curves}

The area between two curves $f(x)$ and $g(x)$ from $x = a$ to $x = b$ is:

$$\text{Area} = \int_a^b |f(x) - g(x)| \, dx$$

\subsection{Volumes of Revolution}

The volume of a solid formed by rotating the region bounded by $y = f(x)$, the x-axis, and the lines $x = a$ and $x = b$ around the x-axis is:

$$V = \pi \int_a^b [f(x)]^2 \, dx$$

\end{document}
